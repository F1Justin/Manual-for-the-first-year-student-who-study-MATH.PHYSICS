\chapter[幻想乡工程技术见闻录 \~{} Fantasy from Triviality]{
\begin{center}
	\Large{\textbf{幻想乡工程技术见闻录} \~{} \textbf{Fantasy from Triviality}}
	\rightline{\large\textbf{——力学专业概貌及基础数学介绍}}
\end{center}}

\normalsize
\addauthor{山舞银蛇 \& 小麻雀}\vspace*{-3em}
\section*{前言}

 {\itshape\small 无论从理科还是工科的视角来看,力学都是十分重要的基础学科。然而就笔者的经验来看,目前国内很多高校要么没有开设力学专业导论课,要么很难起到引导学生认识力学、形成审视力学学科观点的作用,只片面地介绍一些前沿热点问题。此外,力学专业课和数理基础课之间也存在脱节情况,目前国内已有高校认识到这一点,并尝试组织课改。笔者针对以上两点问题,以个人经历及所见所知,参考一些权威的观点和资料,写下了这份手册。

  笔者始终信奉一个观点:有些东西不必刻意追求,当你所做的工作充分时,这些东西就是唾手可得的。在“本科高中化”趋势日益明显的大环境下,只有将所学的内容有机地结合、串联在一起,才能不囿于“内卷”这个小圈子,让学习、科研变得事半功倍。要想做到将所学的专业知识组织在一起,能够把握专业的概貌绝对是一个必要条件,而这一环往往又是大家容易忽视的。

  本手册所涉及的内容都是力学及相关数学的基础内容,不会花费大量篇幅阐述力学的前沿热点问题和比较高阶的数学工具,也不会教授应试技巧,更不是所谓的“速通”或“速成”手册。我们旨在给力学及相关专业的大一、大二新生呈现力学专业的概貌,了解力学学科做过和在做什么,以及对数学基础及其力学应用有一个初步的认识,在此基础上逐渐形成对力学专业的宏观把握。目的是起到补全当前力学学科教育中缺失环节的辅助作用。

  这份手册大致分为两个部分,前一部分对力学作一个概述,从力学的历史出发,介绍力学的主要研究内容和研究方法;后一部分针对的是力学的数学基础,对如何学习和应用力学的数学工具提出建议。当然,对于新生来说,这其中必然涉及一些此前不曾听说过的名词和概念,给阅读增添障碍。我们希望读者能在学习新的力学和数学知识后反复审视一些内容,其实不光这本手册是这样,做学问也应该这样,水平提高后回头看以前不懂的东西,可能会有新的体会与收获。

  八云的魔法书社团(下称“云法书”)活动的基础是东方project系列和数学,然而我们不得不承认在融合这二者这一点上,我们做得并不够好。为了调和二者兼容性的矛盾,笔者产生了用故事的形式呈现这些知识的想法,当然这里的故事并不是此前简单缝合的模式,而是一个剧情、逻辑足够完整的故事。然而由于时间限制,我们无法在今年新生的适应期完成故事的创作,只好先将我们所想的干货内容整理出来,呈现给大家,还望大家能够用充分的耐心等待完整作品的问世!}

\begin{flushright}
	八云的魔法书\\
	新生手册编写-力学及其数学小组\\
	2022.9.29
\end{flushright}
\newpage

\section{力学概述}

我们在中学阶段学习的物理,以力学和电学为重点,特别是力学。我们对滑块、小球、圆形轨道等物理模型真是再熟悉不过了。但是由于所学知识的限制,刚刚中学毕业的我们还不能很好地认识力学。在中学阶段,我们通常会认为力学是物理的一个分支,并且由于数学工具的限制,当时我们只能考察物体以质点模型为基础的简单运动。实际上,力学研究的对象远不止于此,而且力学和物理学已经沿着完全不同的画风发展一个世纪多了,业已取得了相当丰硕的成果,特别是在工程当中获得了广发的应用。

接下来,让我们从力学的发展史出发,了解一下力学的研究内容、分支以及研究方法。

\subsection{力学的发展历史}

力学的内容十分丰富,不是三言两语能够讲清楚的,在概括其发展史时当然也面临这个问题,所以我们接下来的介绍只选择了力学发展中比较重要的主干事件。

\subsubsection{古典时代}

最初,物理学跟力学几乎是同义词。古时候,世界各地的劳动人民就已经在经验上总结许多自然运行的原理,但我们说现代科学的源头在古希腊,因为只有古希腊诞生了形式逻辑与演绎推理,这是数学学科的基础,而一切现代科学又都是建立在形式逻辑、演绎推理与数学描述上的。这里我们介绍一些古希腊时代的著名科学家。

亚里士多德是世界上最伟大的古希腊哲学家与科学家之一,他的研究涉猎广泛,特别是构建了一套影响深远认知系统。尽管亚里士多德的诸多科学结论是错的,却无法掩盖其对现代科学的伟大贡献。

阿基米德是古希腊最重要的数学家、物理学家之一。一般认为阿基米德是刚体静力学和流体静力学的奠基人,他曾经研究过许多数学和力学问题,如球体积公式、抛物线弓形面积、确定物体重心的方法、物体浮力的计算公式、杠杆原理等。阿基米德曾基于杠杆原理给出一个著名的论断,“给我一个支点,我能撬动地球”。此外,阿基米德还制造了许多机械,如螺旋提水机、抛石机等。

古希腊天文学的集大成者是托勒密,他认同“地心说”,并且采用“本轮”和“均轮”学说来解释行星的运动。托勒密体系的观测精度满足了那个时代的要求,同时“地心说”被认为符合基督教的教义,因而这套体系统治了西方世界1000多年,直到文艺复兴时期才被推翻。

这个时期,由于数学工具还比较初等,尽管阿基米德已经有微积分的思想萌芽,但毕竟不成体系,所以力学的研究被限制在比较简单的范畴,主要是静力学和比较简单的运动学。

\subsubsection{力学奠基时代}

文艺复兴时期,古希腊的知识得到复兴,各个学科都呈现出一片繁荣景象,新思想和观点的碰撞,使得人们对客观世界的认识有了很大的进步。达芬奇是文艺复兴时期科学和工程技术领域最重要的人物之一,他提出了许多领先于他那个时代的观点,设计了许多机械图纸,并且还主持修建了诸多水利工程。可以说,达芬奇揭开了那个科学技术快速发展时代的序幕。

伽利略是“近代科学之父”,他开创了系统科学试验与观察的先河,这种实证精神一直延续到今天的科学研究中。伽利略对于力学和天文学的贡献很大,他首次定量地提出“加速度”的概念,奠定了动力学研究的基础;最早给出了惯性定律;给出了参考系、伽利略变换、相对性原理等概念;另外,伽利略还讨论了悬臂梁变形问题。

开普勒基于其老师第谷的天文观测数据,总结并提出了关于行星运动的三大定律:行星绕太阳运行的轨道为椭圆,太阳位于其中一个焦点上;在相等的时间内,行星绕太阳运行的矢径的扫掠面积相等;行星绕太阳运行轨道的半长轴长的立方与运行周期的平方成正比。这三条定律为牛顿最终创立经典力学体系奠定了基础。

最终创立经典力学体系的工作是由牛顿在总结前人工作成果的基础上完成的。首先,牛顿总结出了牛顿三定律,描述物体运动,这些内容大家十分熟悉,就不再阐述了。其次,牛顿研究了万有引力,最终正确地给出了万有引力定律的数学表达式。牛顿的《自然哲学的数学原理》总结了他所做的大部分力学相关的工作,标志着经典力学体系的成熟。牛顿能取得这些成绩的一个关键因素是他发明并采用了微积分的语言来描述,这是与他同时代许多采用几何观点研究力学的学者所不同的地方。微积分这一数学工具极大地拓展了数学能够描述的物理问题的范围,从这里开始,人们广泛地应用微分方程描述物理规律,再加以求解,成为了定量研究问题的一套范式。

这个时期。还有许多其他著名的学者,如惠更斯、胡克、托里拆利等人,他们在经典力学奠基的阶段也做出了重要的贡献,限于篇幅我们不一一介绍。

\subsubsection{力学进一步发展的时代——分析力学}

接下来力学的发展可以用两条线索概括,其一是经典力学理论进一步发展,得到分析力学体系;其二是对连续介质力学,即固体力学和流体力学的研究。

在牛顿之后,经典力学得到进一步发展。当时的科学家们总结出了诸多定律,守恒定律包括动量、角动量(动量矩)、能量守恒定律;达朗贝尔提出了达朗贝尔原理,在非惯性系中添加了“惯性力”,使得能用静力学的手段处理动力学问题;约翰伯努利首次给出了虚功原理的表述。同时,数学工具得到了进一步发展,对于最速降线问题的讨论导致了变分法这一数学工具的产生,这也标志着分析学正式成为数学的一大分支。

尽管经典力学已经建立起来了,但它还是只能处理简单的质点问题,无法处理刚体这样需要考虑物体形状的问题。同时,牛顿的经典力学理论上能处理有约束的问题,但实际问题过于复杂,用牛顿体系处理并不方便。在这种背景下,分析力学建立起来并得到了发展。

欧拉首先研究了刚体的运动规律,扩大了经典力学能够处理的问题的范畴,而且在这个过程中,产生了自由度的概念,即完整的描述一个力学系统的运动情况所需的最少的变量数。例如,若要完整描述平面刚体的运动速度,只需要知道刚体上任意一点的速度(矢量)及其转动的角速度。自由度的概念是分析力学建立的基础,因而十分重要。此外,欧拉给出了极值曲线、等周问题的解答,创立了变分学,并富有创见性地提出“分析学具有重构牛顿力学体系的潜力”。

拉格朗日实现了欧拉的猜想。如果说牛顿力学体系的核心概念是“力”,那么拉格朗日力学体系的核心概念就是“能量”。拉格朗日从牛顿力学出发,结合达朗贝尔原理和虚功原理,推出了(第二类)拉格朗日方程:
\[
	\frac{\mathrm{d}}{\mathrm{d}t}\frac{\partial L}{\partial \dot{q}_i}=0, \quad (i=1,\,2,\,\cdots\,,n).
\]
拉格朗日用广义坐标$q_i$来描述物体的空间位置,其中$n$即为自由度数,这时其对时间的导数——广义速度$\dot{q}_i=\dfrac{\mathrm{d} q_i}{\mathrm{d} t}$的概念就不那么直观了。不过,结合坐标和广义坐标之间的关系
\[
	x_j=x_j\left(q_1,\,q_2,\, \cdots\,, q_n\right), \quad \left(j=1,\, 2,\, \cdots\,, N\right).
\]
以及偏导数的链式法则,表示出牛顿体系中的物理量还是不困难的。式中的 \[
	L=L(q_1,\,q_2,\,\dots\,,\,q_n,\,\dot{q}_1,\,\dot{q}_2,\,\cdots\,,\,\dot{q}_n\,,t)
	.\]是系统动能与势能之差,我们称之为拉格朗日量。

拉格朗日方程与牛顿的动量定理是等价的,就是说,以二者当中某一个为假设前提,就能推出另一个。拉格朗日体系的数学基础实际上就是变分原理,称
\[
	S=\int_{t_1}^{t_2}{L\,\mathrm{d}t}.
\]
为作用量,当作用量达到极小值(变分等于$0$),就能得到拉格朗日方程。作用量这一概念可以推广到一般的演化系统,只要令某一个演化系统的作用量达到极小值,就能得到一个描述该系统运动规律的方程,这就是最小作用量原理。

拉格朗日体系的高明之处在于回避掉了对复杂约束的讨论,使得问题求解得到简化。然而,拉格朗日体系的好处远不止于此,它实际上是一个和牛顿力学等价的力学体系,但等价不意味着二者完全相等,在分析某些问题时,可能二者当中的某一个更有优势。同时,拉格朗日体系具有更好的泛用性,可以应用在不限于力学的更广泛的问题当中。

哈密顿早期的工作是研究光学,但他将其中的数学方法推广到了力学上,创立了哈密顿力学体系。哈密顿对拉格朗日量$L$进行了勒让德变换
\[
	H\left(q_1,\,\dots\,,\,q_n,\,p_1,\,\dots\,,\,p_n,\,t\right)=\sum_{i=1}^n{\dot{q}_ip_i}-L\left( q_1,\,\dots \,,\,q_n,\,\dot{q}_1,\,\dots \,,\,\dot{q}_n,\,t \right)
	.\]
再对该式取变分,就能推出哈密顿方程
\[
	\begin{dcases}
		\dot{q}_i=\dfrac{\partial H}{\partial p_i},   \\
		\dot{p}_i=-\dfrac{\partial H}{\partial q_i} , \\
	\end{dcases} \quad \left( i=1,\,2,\,\cdots \,,n \right)
	.\]
哈密顿推出的方程与拉格朗日的方程和牛顿的运动定律互相等价,但是由于哈密顿的方程具有某种对称性,对于某些问题的研究和求解来说是很方便的,如量子力学的描述就是基于哈密顿体系的。

这段时期,研究人员注意到所谓守恒律实际上是运动方程的某种积分。索菲斯李对微分方程进行研究,将微分方程的解与变换群相联系。诺特进一步提出,作用量的连续对称性意味着守恒。这些思想在日后的物理学研究中发挥着重要的作用。

\subsubsection{力学进一步发展的时代——连续介质力学}

对连续介质力学的研究分为固体力学和流体力学两部分,起初这二者之间没有建立联系,随着研究的逐渐深入,才将它们纳入到一个统一的框架下进行描述。

对流体的研究比固体的稍早。牛顿在《自然哲学的数学原理》之中,就对流体进行了讨论,提出了“牛顿流体”的概念。之后,丹尼尔伯努利、欧拉等人建立了不考虑流体粘性的理想流体的动力学方程,但真实流体存在粘性,所以这部分理论只有在粘性可以忽略时才能实用。不过,理想流体动力学的观点和方法论对后序研究是很有价值的。

纳维考虑了流体的粘性,推导出了不可压缩粘性流体的运动方程,稍后,斯托克斯也从连续介质假设出发推导出了这组方程,这就是著名的纳维-斯托克斯方程(N-S方程)。这组方程的求解十分困难,以至于时至今日,该组方程光滑解的存在性还没有得到证明。

不可压缩流体的典型便是水,这段时期,得益于不可压缩流体的研究成果日益丰富,水力学和水动力学也得到了发展。许多水利工程、水力机械结合生产经验以及理论分析,获得了许多经验和半经验的公式,大大促进了工程学的发展。

20世纪之前,对于可压缩流体的研究则比较少,主要围绕超声速流动和激波展开,做了基础性的探索。

最早的固体力学来自于对杆、梁、板等变形体的变形、强度、稳定性等问题的研究。早期,伽利略考察过悬臂梁的强度问题;稍晚一些的欧拉考察了梁的小挠度变形(横向变形,即垂直于轴向的变形),以及压杆稳定性问题。

随后一段时间,固体力学的研究主要围绕弹性力学展开,这一部分的研究也是比较成熟的。弹性力学理论的先驱是纳维和泊松,纳维提出了各向同性弹性体的平衡方程,泊松给出了泊松比的概念。弹性力学理论的奠基者是柯西,他引入了应变、应力的概念,讨论了应力应变之间的关系,讨论了平衡微分方程和边界条件的提法,这些内容是线弹性力学的基本内容。

建立起弹性力学相关的基本概念后,学者们研究了一些弹性模型的解法。圣维南提出了逆解法和半逆解法,是弹性力学求解的重要手段;以及圣维南原理,描述了弹性体的局部效应,对于问题求解很有好处。柯希霍夫(基尔霍夫)研究了薄板的理论。瑞利总结了弹性体的振动和声学问题。勒夫总结了前人关于弹性力学的工作,并进一步发展,讨论了薄壳问题,同时还研究了弹性波理论。穆斯赫利什维利致力于用复变函数的手段求解弹性力学问题,是计算力学发展之前的理论求解的高峰。

弹性力学的发展完善了对杆、梁、轴等基本结构件变形和强度的研究,获得了一些简单且具有工程实用性的结果,这一部分当属材料力学的范畴。同时期,工程上开始结构的变形问题,最典型的是桁架结构和连续梁,并发展了一些求解的方法,这当属结构力学的范畴。此外,由于工程实际的需要和工程事故,人们开始关注疲劳和断裂现象。

\subsubsection{近代力学}

20世纪初,经典物理学面临着两大问题,其一是黑体辐射问题,由此产生了量子力学,其二是探讨光速不变的问题,由此产生了相对论。从此开始,物理学和力学之间出现了比较明显的界限。

量子力学关心原子、分子尺度上基本粒子的性质,相对论则深入地考虑了我们所处宇宙的时空背景。这二者都是在不同情况下对经典力学的修正,经过适当的退化,它们可以分别得到经典力学中的结果。同时,经典力学能够足够精确地描述宏观、低速下的物理问题,而且这些问题还没有被完全解决,所以经典力学不仅没有走到尽头,反而得到进一步的发展,直至今天,形成了目前的力学这一学科。

时间来到20世纪,力学比较明显地分化出了“应用派”和“理论派”。这是由于,一方面理论物理的发展已经远远超过工程实践的能力,于是诸多工程内容需要依靠力学理论指导进行;另一方面,即使是经典力学中尚有诸多问题没有得到回答,有些知识更需要建立一个完善的体系。这种分化也不是20世纪才出现的,事实上早已有之,总有一些理论是比较超前的,暂时找不到它的用处,也有一些理论是为了解决当前的问题而提出的半经验或经验的模型,这就分别对应“理论派”和“工程派”。不过,这二者之间不是完全对立的关系,理论与实践是相辅相成的,只是二者之间分化的趋势到20世纪比以前更加明显了。目前,将力学与工程相结合是主要的趋势。

20世纪初这段时间的力学发展了以下主要内容:

一般力学中的许多问题最终都归结到求解常微分方程组中,不过解析求解一般的常微分方程组是困难的,于是人们便去研究方程的性质和行为。在研究非线性微分方程的过程中,陆续发现了周期解、非线性振动、分叉、混沌等现象,促进了微分方程定性理论的发展。%%%%配图

流体力学部分继续对N-S方程和湍流问题进行研究。雷诺将速度写为平均速度与脉动速度之和,将压强写为平均压强与脉动压强之和,得到了平均意义下的N-S方程,后人称之为雷诺方程。普朗特、冯卡门、泰勒、周培源、柯尔莫哥洛夫进一步做了一些工作,开创了湍流统计理论的研究。另一边,飞机的发明刺激了对机翼升力问题的研究,普朗特、兰开斯特、芒克等人合作建立了“升力线”理论,对翼型设计起到了指导作用。

固体力学部分,人们已经将弹性理论中的大部分线性问题研究清楚了,这时开始考虑梁、板、壳等结构的稳定性,以及板壳的一般理论等。塑性力学、断裂力学以及疲劳问题等研究也已开始进行,就是说,这段时间的研究以及不再限于线弹性问题,而是向更多的非线性问题发起冲击。此外,麦克斯韦等考虑了黏弹性体模型,进而引发了人们对一般连续介质的性质的探讨,并最终形成了连续介质力学,并且促进了理性力学的复兴。

此外值得一提的是计算机的发明导致了计算力学这一分支的产生。20世纪初,人们已经需要求解大规模复杂工程中的力学问题,但庞大的问题规模时代靠人力解决这些问题是极端困难的。计算机的问世和发展,强化了人类的计算能力,使得求解大规模力学问题成为可能。计算力学就是这样一门借助计算机解决力学问题、分析力学性质的学科,它处于力学、数学、计算机三门学科交叉的位置。计算力学当中的方法已经能够比较好地解决大部分线性问题,而对于数学本质是非线性的问题,如何用计算力学的手段去解决和分析还是很有挑战性的课题。

\subsection{力学的研究内容}

概括地说,力学是\textbf{研究宏观低速的物体机械运动规律的学问}。这一句话中包含了两个限定以及研究的对象。

\textbf{宏观},指的是忽略量子效应,而不一定是肉眼可见才叫宏观,有时,力学也会研究微纳米尺寸的问题,而不计量子效应的影响。\textbf{低速},指的是忽略相对论效应,天体力学能够解决太阳系范围内很多天体运动的问题,但在更大尺度的宇宙环境中,可能要考虑相对论效应的影响,这已经超出力学的研究范畴。总之,除了个别的极特殊情况,我们所讨论的力学都是经典范畴的。

\textbf{物体},其含义可以按照基于某种假设将实际物体抽象成的某种模型,再对这个模型加以研究。这些模型无外乎以下几种:质点、刚体、连续介质。

我们在中学阶段就学过了质点模型,这种模型假设忽略物体的尺寸大小,将物体视为有质量的点,例如,在考虑行星绕恒星运动时就会应用质点模型。

刚体则考虑了物体的尺寸、形状,但仍然认为物体是刚度无穷大的,也就是不会发生变形,准确的说法是“在物体上任取两点,这两点的距离是一个恒定值”。这种刚体模型相较于质点模型更加常见,杆、齿轮等机械构件在运动时,不可以忽略其形状因素,这时可以将它们视为刚体。

连续介质是一个相当宽泛的模型,它可以涵盖多数固体、液体、气体。其最显著的特征是要考虑物体的变形。我们将这种模型称为“连续介质”的原因是,该模型的基本假设是“连续性假设”,即材质相同的一块物质是无限可分的,这样就可以应用微积分、场论等分析学工具对物体的变形加以研究。再根据物体变形的性质(可以理解为内力与变形及变形速率的关系)不同对其进行分类,粗略地讲,按照物体能否有效地抵抗剪切力将物体分为固体和流体,再按照流体是否可发生压缩变形分为气体和液体,按不同类别分别加以研究。

\textbf{机械运动}是描述物体的空间位置随时间变化的过程,大致可分为刚性运动和变形。所谓刚性运动,即是指质点和刚体可以发生的机械运动,如平移、转动;而变形是物体的形状在外界作用下发生了变化,如弹簧的形变、弹性体的形变等。

我们首先对力学的研究范围做了一个大致限定,也了解了力学中的一些基本模型和概念。然而在某些必要的时候,可以适当放宽这些限制,力学的研究对象也可能随着科学技术的进一步发展而得到扩充。
目前,力学学科的二级分类大致就按照模型不同,分为一般力学、固体力学和流体力学,此外还有更加面向工程应用的工程力学和一些其他分支。下面,我们具体到目前力学学科的主要分支当中,看看不同力学分支的主要研究内容。

\subsubsection{一般力学与力学基础}

一般力学大致是动力学与控制的同义词,它的研究对象是宏观的离散力学系统和经典力学的一般原理。力学基础的对象比较广泛,但仍以离散系统为主。

\paragraph{天体力学}

天体力学是天文学和力学之间的交叉学科,研究对象小至太阳系内的天体,大至恒星系。其又下含几个次级学科,包括摄动理论、定性分析、数值计算、天体形状与自旋理论等,以及其他一些相对独立的课题。

\paragraph{刚体动力学}

刚体动力学研究刚体在外力作用下的运动规律。它是计算机器部件的运动,舰船、飞机、火箭等航行器的运动以及天体姿态运动的力学基础。主要的次级研究内容有陀螺力学、转子动力学等。

\paragraph{分析力学}

分析力学通过用广义坐标为描述质点系的变量,运用数学分析的方法,研究宏观现象中的力学问题。早期内容包括拉格朗日力学和哈密顿力学,后来又对有约束系统做了讨论。目前理论体系已经比较完备,主要考虑其具体应用。

\paragraph{运动稳定性}

运动稳定性研究物体或系统在外干扰的作用下偏离其原有运动后返回该运动的性质,对于运动稳定性的研究大都出于工程技术需要。奠定运动稳定性相关理论基础的是庞加莱和李雅普诺夫,特别是李雅普诺夫开创的分析方法已经在诸多领域中得到应用。

\paragraph{非线性振动}

恢复力与位移不成正比或阻尼力不与速度一次方成正比的系统的振动都算是非线性振动。线性振动理论已经发展得比较完善,但在实际问题中,总有一些用线性理论无法解释的现象。一般来说,线性模型只适用于小运动范围,超出这一范围,按线性问题处理就会引起较大误差。

\subsubsection{固体力学}

固体力学是力学当中成形较早、理论性和应用性都比较强的一个分支。它主要研究变形体在外界影响下内部发生的运动、变形等规律。固体力学当中的线弹性、小变形理论已经发展得相当完善,目前研究的方向多是非线性问题及随机性问题,典型的非线性问题有大变形、稳定性、断裂、疲劳、冲击等。

\paragraph{材料力学}

材料力学主要研究简单结构件的变形、稳定性和强度问题。广泛地说,断裂、疲劳、强度、蠕变、试验测定等都可归入到材料力学的范畴。实际上,材料力学并不能算是一个独立的研究方向,它其实是在弹性力学获得结果的基础上,引入合理的假设,得到足够工程实用的结果。因此,材料力学更像是面向工程师的弹性力学简化版,或者作为入门固体力学时的学习材料。

\paragraph{结构力学}

结构力学研究的是杆、梁、板、壳等工程基本结构及其组合的变形和受力。结构力学由材料力学和弹性力学发展而来,主要研究内容有结构静力学、结构动力学、稳定性理论、断裂和疲劳。其中,结构静力学的发展比较成熟,并且由其中的“位移法”产生了计算力学的雏形。结构动力学研究的是在动态外载作用下结构的响应,主要围绕结构振动展开,目前比较关心的主要是复模态理论和主动控制问题。稳定性理论、断裂和疲劳则是关注工程结构,而非一般的弹性体。

\paragraph{弹性力学}

弹性力学是固体力学的基础,它研究一般弹性体在外载作用下的内力、变形情况。线弹性力学的基本理论早已建立,一些典型问题的解也已给出,也在此基础上分化出了材料力学和结构力学。目前,弹性力学有两个方向,一个是与其他学科相互交叉,以研究更多的模型;另一个是继续深化弹性力学的数学理论,研究非线性问题。

\paragraph{塑性力学}

弹性体在受外力较小时,发生弹性变形,满足广义胡克定律,当弹性体受力超过其弹性极限后,其变形不再满足广义胡克定律,同时也会发生不可恢复的变形,这就是塑性变形,而塑性力学自然就是研究弹性体发生塑性变形时,其变形与内力与外力之间关系的学科。塑性力学的主要研究内容有屈服条件、塑性增量理论和塑性本构关系。

\paragraph{断裂力学}

断裂力学是研究材料断裂失效和裂纹扩展规律的学科。该分支早期研究了裂纹扩展准则、裂纹尖端应力场分析、应力强度因子的计算方法等,以及如何将其应用到工程中去。目前的一些研究方面有:智能材料(压电、铁电等材料)的断裂与损伤、微纳米尺度断裂、非均质材料的断裂与损伤、断裂和损伤的试验与工程应用。

\paragraph{复合材料力学}

复合材料是由两种或以上不同材料经过特定工艺复合而成的新型材料,其性能往往比传统材料更优,且具有良好的可设计性。宏观来看,复合材料最典型的特征是各向异性,因此复合材料问题往往也具有一定的非线性。目前,复合材料力学主要研究以下几个方面:复合材料基础力学理论、纤维增强复合材料的细观力学、特种环境复合材料力学性能模拟表征与优化设计、功能梯度复合材料。

\subsubsection{流体力学}

流体力学主要研究流体和气体在外力作用下发生的质量、动量、能量传输。

\paragraph{理论流体力学}

流体力学的基本方程已经建立起来,但是其求解是十分困难的。流体力学的基本理论有理想流体力学、粘性流体力学、空气动力学。目前只研究清楚了一些特定流动形式,一般的流动、湍流现象还是很棘手的。此外,还研究非牛顿流体及其应用、孤立波等问题。

\paragraph{工程流体力学}

工程流体力学包括很多分支,如水力学、气动力学、渗流力学、多相流等问题,所研究的问题往往具有较强的非线性,直接求解十分困难,往往要借助新的数值手段和实验手段。

\subsubsection{其他分支}

力学领域还有许多分支,如计算力学,可以与上述几个大的方向相结合,去研究对应领域中的计算、仿真方法;再如工程力学,关心实际工程问题中的力学问题,注重应用各种力学手段去解决工程技术、制造当中的问题;又如理性力学,通过基本公理和数学原理推导力学知识体系,力求严密性。此外,还有许多新兴的领域,属于力学与其他学科交叉,例如生物力学、机器人动力学、地球力学、岩土力学等等,限于篇幅不一一介绍。

\subsection{力学的研究方法}

对于一般的自然科学来说,研究手段很多,但是力学的主要研究手段只有三项:理论分析、试验测试与数值方法。

\subsubsection{理论分析}

理论分析是力学学科的传统分析手段。从假设出发,利用数学工具,建立模型的控制微分方程及边界条件,或者是积分方程。如果能求得解析解,那么就将其解析解求出来之后分析,也可以直接进行分析。

理论分析是力学与数学联系最紧密的部分,一方面,数学当中的定量分析对于准确描述力学模型而言是极为有用的;另一方面,对力学模型的研究也促进着数学的发展,这一点从牛顿建立微积分来研究经典力学起就是这样了,直到今天。

\subsubsection{实验测试}

实验与测试是考察我们所建立力学模型究竟是否符合真实世界中物体运动规律的手段。目前,实验手段是多种多样的,机械测量、光测、电测等方法,并形成了专门研究实验方法与手段的实验力学。随着新技术的发展,还有新的测量手段与方法不断涌现出来。

\subsubsection{数值方法}

数值方法的主体就来自于计算力学,源自于早期的有限差分法、瑞利-里兹法、伽辽金法,随着计算机的发展而兴盛起来,不断出现应用于各种问题的计算方法。目前,力学当中几乎各个领域的研究都离不开数值方法,数值仿真更是提供了一种节约成本的研究手段。对于一种数值计算方法,只有当其可靠性和效率都得到保证,才能算是成功的。一些比较成熟的数值仿真方法也已商用,在工程实践中发挥重大作用。

另外,近年来以数据驱动的研究方法逐渐被力学研究者采用,有人主张将其归类为新的研究手段,但这里姑且也将其划到数值方法之中,因为这种方法也是基于数值处理的,且离不开计算机和人工智能技术的发展。数据驱动指的是从初始的数据或观测值出发,运用启发式规则,寻找和建立内部特征之间的关系,也指基于大规模统计数据的处理方法。力学当中主要使用的还是机器学习手段。
\newpage

\section{专业数学培养}

数学是研究现代科学与技术的基本语言,力学更是工科之中与数学联系较为紧密的学科。掌握足够的数学知识是十分必要的,其意义不仅仅是为了应付考试,更在于掌握数学知识有助于准确地把握力学问题的本质,以及快速入门未曾研究过的专业问题,等等。学好力学的核心,除了培养力学专业的思维方式与直觉,最重要的便是要学好数学知识了。

不过,学习任何一门学科都是需要付出很多精力的,特别是数学是一门对抽象思维能力要求较高的学科,学习起来的确有一定难度。所以我们最关心的问题自然是:如何学好数学?这其实是一个很大的问题,我们这里只能根据从笔者有限的经历,对学习的总体方法论和知识树作一简介,姑且算是给读者提供大致的理论指导。而读者则需要根据自身实践,对我们提供的指导思想做适合自己的调整,找到最适合自己的学习节奏。

此外,力学的学习与数学的学习具有一定的相似性,有些方法是相通的,但我们着重介绍基础数学知识,对力学专业内容不做展开,还请读者在后续学习中自行体会与摸索。

\subsection{总体方法论}

\subsubsection{培养自学能力}

大学教育与中学阶段的最显著的区别是,授课教师不会像以前一样手把手地教,把所有的知识点都嚼碎之后喂给我们。实际上,大学设置课程的质量参差不齐,如果只是一味地跟着授课教师亦步亦趋,往往是学不到多少东西的。同时,在以后的研究、工作之中,很多事情也都需要自己去想办法解决。所以培养自学能力就十分重要了。那么,如何培养自学能力?

\paragraph{兴趣是最好的老师}

作为一名理工科学生,在学习过程中会用到很多数学。微积分、线性代数等等不一而足,然而对于力学系的同学来说,计算能力和建模应用是首位的,逻辑上的推理以及严格性的证明往往是被我们作为学习数学辅助手段来使用的。

力学系的同学使用数学工具的过程中,常常会对自己使用的数学工具产生疑问,这样做是否是合理的?为什么会是这样?为什么有的地方满足交换律,到了别的地方就不满足了?同样是两个方程确定的偏导数,为什么在表达顺序上会出现异常的规律?$x$和$y$,究竟在哪里等价,在哪里又不等价?深入的思考带来的疑问,驱使着我们去发现未知的数学规律,而在这个过程又会带来极大的愉悦和满足。从工科的角度学习数学就是这样一个过程,从使用中产生疑问,在解答疑问的过程中产生新的想法。大学毕业以后,有些专业课有可能以后很难再接触到,真正珍贵的是自己独立思考的过程,和灵光一现,再到解决疑问的愉悦感。

在大学里,建立良性的循环是非常有必要的。什么是良性的循环?不妨先看看什么是恶性的循环。假如你有一天熬夜到很晚,第二天跟不上老师,下课后作业不会,费尽心思搞懂又是深夜。在这个过程中一点点磨灭掉自信心,从此开始摆烂,直到挂科,退学等等事情的发生。规律的作息是保证良好心态,形成良性循环很重要的一部分。而良性循环就刚好相反,比如学习——兴趣——思考——学习的过程,这样的循环对于坚持学习是非常重要的,学习的开始往往是从兴趣开始的,这也是为什么兴趣是最好的老师。

\paragraph{资料与信息检索}

\begin{enumerate}
	\item \textbf{善用图书馆}。如果将图书馆只是当作自习室,那实在是暴殄天物。对于一个大学生来说,图书馆有着最直接也最好用的海量图书资源。你可以在图书馆借到大量的、侧重点各不相同的同一类书籍。有些大学的图书馆会有许多国外的或者国内已经绝版的、甚至是独一无二的优质资料。不夸张的说,图书馆至少能够满足一般人在本科阶段大多数的资料需求。同时,善用图书馆的检索功能,掌握书籍的检索规律,还能锻炼资料检索的能力。
	\item \textbf{电子资源的检索}。Zlibrary、知网、Sci-Hub、学校图书馆的线上借阅、各种不知名的小网站、百度网盘等等都藏有许多资源,这种资源检索往往需要对所研究的领域有一定了解。在入门阶段,不妨应用百度、知乎、Google、CSDN等检索平台,对于某些领域来说能够满足快速入门,获得一个初步印象的需求。
	      关于更多的具体的数学检索平台,可参见 \href{http://home.ustc.edu.cn/~yx3x/USTC/ustcmathplan3.pdf}{USTC基础数学修课指南}。
	\item \textbf{充分调动人脉资源}。很多时候,我们缺少的往往是获得关键信息、资料的途径,这些内容不是通过阅读已有书籍或电子资料就能获得的。这时候,要积极地向身边或线上了解相关领域的朋友请教,他们手中的资料、他们对于该领域的见解,都是相当宝贵的信息,这要比自己探索方便快捷得多。但是,如果有志于独立研究,也不要过分依赖于他人的帮助,因为总有一天会面临无人可问的情形。
\end{enumerate}

\paragraph{正确处理习题,形成自我评估}

对于习题的争论,历来分成两派,一派主张“题海战术”,另一派与之相反。究竟应该选择哪一种,取决于学习动机。

对于应试来说,“题海战术”确实是行之有效的方案。但对于一般性的学习,我们不建议直接捧着习题集做,却也不是少做题。在反对“题海战术”的思潮中,也有另一种极端,即动笔很少,这也是不可以的。题必须要做,只是最好要有选择性地做、精做。做题的主要目的有两个:一是通过操作实例,积累经验与直觉,所以像《吉米多维奇》这样习题重复度过高的参考书并不值得大家花时间去大做特做;二是检验学习程度,判断目前的水平是否可以继续深入学习,只有打牢基础才能钻研更深入的知识,前置知识掌握得不牢固,学习后续知识就容易出现困难。

如何做题也是有讲究的。首先,尽量选择有参考答案,最好是有详细解答步骤的题目去做。其次,在做题时,独立思考是很重要的,但也不要受“独立思考”这种说法的束缚,题目中某个关键步骤如果想了很长时间也没想出来,就应该直接去看答案了。我们的目的是掌握方法,把答案看懂(而不是囫囵吞枣)当然能够掌握方法,没必要有什么心理负担,我们如今所学的知识都是前人经过数十年、数百年浓缩而成的,把所有细节全都背下来,既是不现实的,也是没必要的。

\paragraph{快速入门学科,不必万事俱备}

在学习和科研的过程中,我们常常有快速入门一个学科/领域的需要。要想快速入门一门学科,最好还是直接看问题,从问题下手往往是最快的,因为这时有足够的动机。不必管解答,无论能否看懂,起码都会先有一个印象,以及自己思考的过程,这样就可以算得上入门了,入门之后再去考虑其中的细节问题。

比较忌讳的是为了一个领域/专业课,去大学特学前置知识。一来这样效率低;二来准备得完全充分也是不现实的,在学习和研究的过程中经常会出现新的概念,现用现学才是常态;三来是不同专业的侧重点不完全一样,如果学了某样“前置知识”,到最后发现在专业课的实际应用中并没有用到那么多知识,就有点得不偿失了。

此外,快速入门的另一个前提是要有广泛的知识储备,也就是说在一定程度上当一个“名词党”是有好处的。不管怎么说,知道有一个名词比不知道强,在前期调研时至少有一个大致方向。第一次接触某一个名词时,不一定要对其有多么深入的理解,哪怕仅仅是形式上接受它也未尝不可。顺便一提,乱用名词是工科人的特色,不得不品尝。

\paragraph{学数学是一个积累的过程}

很多人可能会认为,理工科类型知识的学习只要捋顺逻辑即可,无需进行记忆。事实并非如此,倒不是说我们需要去背公式、记习题,而是要积累观点、思想方法和典型例子。“数学成熟度”是一个比较玄学的指标,看不懂书的时候,可能有人会对我们说:“这是数学成熟度不够导致的”。“数学成熟度”大概指的就是我们习惯一套数学语言的程度,我们在数学上做积累的目的就是提高“数学成熟度”。

如何提高“数学成熟度”呢?凡事没有捷径,唯一的捷径就是抛开杂念去做。数学的学习,分为输入和产出的过程。看书、做题、上课是输入,与人交流、整理笔记、编写讲义、写小论文等是输出的过程。数学思维的建立不是技巧的堆砌,而是积累大量的数学观点和思维方式,能说出定理和定理、学科与学科之间的关系。循序渐近,多看、多写、多想,才能逐渐提高“数学成熟度”。这是一个或长或短的积累的过程,不必急于求成,踏实去学才能厚积薄发。

\paragraph{一些学习方法}

在学习过程中,还有一些具体的方法:

一定要充分发挥主观能动性,我们的目的是学知识,不是折磨自己。这意味着可以先通读学习材料,先对知识形成一个整体印象,遇到不懂的一个点可以选择性地跳过,待通读若干遍之后,再精读,去认真考察其中的细节。这也是一般阅读文献应该采用的思路。

抄书是自学的有效方法。它可以是选择性地摘录重点,也可以是事无巨细地誊写,抑或是学习一遍后的总结、归纳。这种方法看似笨重无比,实际上是能稳定获得较大收益的办法,抄书能辅助你集中注意力,完成构建知识体系的过程。

若是比较关心数学的力学应用,则不必限于技术性细节,更多地关注计算与应用。可以选择性地忽略一些严谨的证明,暂时接受一项结论,在具有一定理解的前提下考虑如何去计算和应用,通过不断地计算和应用反过来加深对概念的理解,这既包括对数学概念本身的理解,也包括概念及相关的计算方法可以被应用在哪些物理和力学模型下,在有进一步需要的情况下,可以考虑花些时间仔细地研读证明过程。

若是出于兴趣,或是为了学习一些高级的数学理论和工具,请一定打好前置知识的基础,关心问题的处理思想与方法,关键性的推理与证明都不要放过。很多进阶的概念和方法是从比较基本的问题中提炼、概括出来的,对基本背景问题有足够的了解,能更好地理解提出某项概念或定理的动机,以及了解其研究思路,处理一些问题时能有一个大致的思路。

\subsubsection{形成认识与对待数学的观点}

之所以单独开设这一个板块,是因为笔者发现即使到了本科毕业的时候,仍然有许多专业内的同学没法分清数学和物理/力学之间的区别。这源自于我们中学阶段对数学基础的忽视,而只是简单归纳“性质”、“应用”造成的恶果,如果延续这种“一叶障目,不见泰山”式的思维进行学习,是走不长远的。很多人学了数年数学和物理,只能看到“数学和物理当中都有很多定理、公式”、“都需要做大量计算”,却难以分辨出二者之间除了研究对象以外的区别,只是觉得数学似乎更抽象一点,物理/力学更实际一点,实际上这是浅显的、片面的。

最重要的区别在于,数学不是自然科学,而是形式科学,而物理/力学都是自然科学。自然科学是一门以观察和实验的经验证据为基础,对自然现象进行描述、理解和预测的科学分支,就是说自然科学要取之于实践,终究也要用之于实践。形式科学并不关心真实世界与理论之间的联系,而关注以定义和规律为基础的形式系统的性质。

数学作为形式科学的属性决定了,在数学当中,只要推理是正确的,那么得到的结果就一定是正确的,不存在任何更改的余地,一旦需要修正,一定是推理出现问题,或者需要相应修改推理的前提。而在自然科学中,所有理论都是通过观察、归纳,总结出基本定律之后获得或再推理获得的,观测和归纳出现问题,都可能导致理论的修正。同时,数学也可以作为一种语言对自然科学进行定量描述,构建数学模型对实际问题加以研究,正是这一原因容易使人们看不清数学与自然科学之间的关系。

时常有人对数学发出诘难:“数学有什么用?”,然而数学作为形式科学本就不必考虑实用性,以及现实世界是什么样子的,它可以是纯粹的思维游戏,这无可非议。可是,数学又不能完全脱离实际。这话看起来有些矛盾,但一切抽象的知识和问题最初都一定来源于实践,只是经过不断地归纳、总结、抽象、一般化的过程中,我们可能已经看不出问题原来的模样。

不过作为力学专业的学生,我们更多地是把数学作为描述力学问题的语言与解决力学问题的工具。如果拿作战做一个比喻,我们的力学原理和直觉就像是战略与战术思想,指引我们的研究方向,而数学就像是支持作战的后勤保障,支持着研究进行以及这场研究能够走多远。

抽象也好,实用也罢,研究抽象数学理论和研究数学理论与现实情况之间对应关系的数学家都应该得到我们的尊重,他们从事的是人类头脑所能进行的最伟大、最艰深的活动之一。

\subsubsection{正确处理数学与力学专业的关系}

\paragraph{注意学习数学的动机}

在确认学习之前,一定要明确学习动机。学习数学应当以力学专业课程为导向,兴趣爱好为辅,无需过早地接触过于抽象的数学。我们当然支持出于兴趣学习一些高级的数学工具,而且这也是有好处的,但是一定要避免在这个过程中形成眼高手低、浮于表面的坏习惯。

我们指出,所谓高级工具本质上更接近对“相对初级”的理论的归纳、抽象、推广,有些高级工具需要学习到足够深入之后才能拿来解决一些问题,这对于力学学科来说的代价很大,务必做好取舍。也有些高级工具压根就不是为了解决现实问题服务的,或者说至少不是为了你感兴趣的领域准备的,这时也需要考虑清楚学习成本。

再者,务必端正深入学习的动机。中学阶段的应试教育培养出了一种奇怪的思潮,即所谓“应用高级数学工具对初级问题进行‘降维打击’”。经验表明,高级数学工具往往并不会直接产生“降维打击”的效果,它起到的往往是提供思路的作用,甚至可能不起作用,换句话说,通过高级工具实现“降维打击”的收益至少不明显大于打好基础,这并不是高级工具应有的作用。熟悉了高级工具之后,应该做的是去了解更广泛的数学、应用知识,在反复使用中熟悉、理解、内化,将“高级工具”变成你的“初级工具”,这才是正确的打开方式。

另外,如果出于应急所需,例如一门专业课要用到某样数学工具,这时也不建议直接去找一本数学理论性很强的书去看。这样不仅见效慢,而且可能数学方向与专业所需的关注点并不一样,更可能打击自信心,为学习徒增障碍。更好的做法是边学专业内容边补数学工具,以及找一些有附录讲解这些工具的同类书籍,在有了基础知识和专业知识后,可以考虑再看相应的数学理论。

\paragraph{注意学习数学应该深入到什么程度}

力学学科与数学的联系是相当广泛的,乍一看学习力学需要特别多的数学,但实际上所有内容的基础都是微积分与线性代数,这是一切学习的根本与下限。出于不同的学习动机,学习的深入程度自然也是不一样的。

如果只是为了应付考试、科研,那掌握最基本的微积分和线性代数可能就足够了。但是,这时去备考或阅读文献可能会比较痛苦,因为很多数学概念是有更深厚的数学或物理/力学背景存在的,如果不能把握这一点,那么所做的工作可能会显得比较破碎、凌乱,不容易直击问题要害。

如果为了在科研中取得一定成果,或者对数学有一定兴趣,那么至少本科数学基础课都应该学好。我们后边将会介绍,本科阶段的基础数学课都是有用的,会在某些力学课程中用到,只会微积分和线代的话,就算能看懂,理解起来也会存在很大障碍。

如果为了专门处理某类问题,或者对某一数学领域兴趣浓厚,则应该广泛地参考课外学习资料。这个阶段不必限于培养方案中的设置,根据自己的需要、兴趣广泛地学习即可,但也要注意深入的程度,没有必要学习相当抽象的理论,仍然以计算和应用为主。(除非是确实对抽象的数学理论感兴趣,但若是如此,建议自行平衡好精力)

\paragraph{注意寻找与建立数学与力学之间的联系}

学过相应的数学后,一定不要束之高阁,而要寻找和思考这些数学工具可以用在哪些地方。知识只有在不断运用之中,才能得到深层次的理解,长时间不用自然会遗忘。具体的联系我们将在具体的数学科目中介绍。

\subsection{基础数学培养}

前一节业已提及数学的重要性,以及如何把握数学知识。本小节将介绍大学内最基本的数学知识——微积分及线性代数,以及其相关知识的基本外延。微积分和线性代数是描绘一切理工类学科知识以及进阶数学知识的基础,其重要性不言而喻,一定要给予足够重视。然而,如何有针对性地学好这两门基本数学课程,使之与日后专业知识的学习相适配,不同专业又有不同的侧重点,这里我们只勾勒力学类学生应该需要把握的大体框架。

具体的介绍之前再次强调,我们所讲的内容不以应试为导向。

\subsubsection{微积分}

\paragraph{主体知识介绍}

微积分是研究微分和积分是数学分支。粗略地说,微分讨论的是“变化率”,积分讨论的是“累积”或“求和”。在力学上有两个经典的例子分别给出了研究微分与积分的动机:为了计算物体在运动过程中某一点$M_0$处的瞬时速度,再取一点$M$,能算出这段路程上的平均速度,当$M$足够接近$M_0$,这个平均速度就能近似描述瞬时速度,这个过程就是微分;为了计算物体在变力作用下运动过程中所做的功,先将运动轨迹分成若干段,在不同段上用分别用一个常力代替变力计算做功功,随后求和,当运动轨迹划分得足够细致时,计算的结果就近似于真实的做功大小,这个过程就是积分。

学习微积分首先要有数列和函数的概念,函数研究的是变量之间的变化关系,数列可以看成定义域离散的函数。在正式学习微积分之前,要对基本初等函数有一个了解,知道其基本性质、函数图像。

一元微积分研究的是一元函数的微分与积分,其基础是极限理论。

一元函数微分部分的核心技巧与内容有两块:Taylor公式与微分形式不变性。Taylor公式是描绘可微性足够好的函数在某点附近性质的有力工具,所以要对各种基本初等函数的Taylor公式相当熟悉,同时,它也是对“等价无穷小”这一概念的更精确的表述,能够处理“极限中函数什么时候可以做加法”的问题。微分形式不变性是能将复合函数链导法则、隐函数求导等内容串到一起的,从这里出发能将前边所学的很多东西联系起来。

关于微分需要多说一点。有些教材上对于微分的处理很模糊,较好的教材会说“微分是函数变化过程中的线性主部”,我们在此阶段,不妨就把微分看成是差分的极限。在处理一阶导数时,可以在形式上将其视为$\mathrm{d}y$与$\mathrm{d}x$的商,但是高阶导数则不行,也因此“反函数的导数是原本函数导数的倒数”这一奇怪的结论(实际上,基于微分形式不变性就能求解反函数的导数)仅能对一阶导数成立。在后续的学习中,也能看到“将微分看作某个线性空间下的基”这种观点,那是后话了,多数时候将微分看成差分的极限是够用的。

一元函数积分的重要内容是分部积分,以及Newton-Leibniz公式。不定积分是导数在相差一个常数意义下的互逆运算,但正如减法引入负数、除法引入有理数,初等函数的不定积分也不一定是初等函数。事实上,绝大多数的初等函数的不定积分都不是初等函数,所以能够掌握一些基本的、常见的不定积分技巧,包括换元、分部积分即可。而分部积分是很重要的,这个技巧在涉及到能量的方法以及变分法中是常用的。Newton-Leibniz公式的精髓不只在于建立了函数及其导数的关联,更重要的是建立了区间和区间边界的联系,这一点在多元积分当中的各种积分公式中能够看到。

多元微积分研究的是多元函数的微分与积分,以及一些场论的知识。

多元函数微分部分,首先要掌握$n$维欧氏空间$\mathbb{R}^n$中的诸多基本概念(拓扑性质),这些基本概念在后续学习中还会用到。最重要的概念当属偏导数与全微分,偏导数只描述函数关于某一个自变量的变化情况,而全微分反映各个自变量的影响,因此全微分比偏导数要强很多。类似一元微分学部分,全微分也具有形式不变性,同时也能搞定复合函数求导以及隐函数求导。这里可能要注意不同学科的习惯不同,比如力学中,今有一个函数形式为$L=L(q(t),t)$,若要求其对时间$t$的全导数,往往会写成以下形式:
\[
	\frac{\mathrm{d}L}{\mathrm{d}t}=\frac{\partial L}{\partial q}\frac{\mathrm{d}q}{\mathrm{d}t}+\frac{\partial L}{\partial t}.
\]
形式上看,这种记法符合全微分形式不变性,但数学上可能记之为$L=f(q(t),t)$
\[
	\frac{\mathrm{d}L}{\mathrm{d}t}=\frac{\partial f}{\partial q}\frac{\mathrm{d}q}{\mathrm{d}t}+\frac{\partial f}{\partial t}.
\]
没有哪一种写法是错的,只要注意不同的习惯约定即可。多元函数微分学常用一些空间解析几何的内容,在此之前最好先学习一下相关内容,知道常见空间曲线、曲面的解析式。另外,从这里开始逐渐接触到一些涉及到梯度场的知识,由于数学工具限制,在这一阶段不会太过深入,知道一些关键结论和初步应用即可。

多元函数积分与场论部分,首先要熟悉基本的累次积分、二、三重积分的计算,以及不同坐标系下积分微元的关系。随后是两类曲线/曲面积分,第一类曲线积分主要注意选择合适的参数将弧微元转化成关于参数的一重积分,第一类曲面积分主要注意选择合适投影面将面微元转化成关于两个坐标的二重积分。第二类曲线/曲面积分是矢量积分,一方面注意其与第一类曲线/曲面积分的联系,另一方面注意Green公式、Gauss公式、Stokes公式的运用。此外,此处会进一步涉及散度场、旋度场的内容,需要熟知梯度算子的基本计算,以及“有势无旋、有旋无源”等常用的结论。

一般的微积分教材还会有两个专题,一是常微分方程,二是无穷级数。微积分课程中所讲的常微分方程是比较有限的,最重要的当属常系数线性微分方程和Euler方程,我们重点关注其求解。无穷级数中最重要的是Taylor级数和Fourier级数,Taylor级数部分注意运用运算规则推导常用的级数,Fourier级数部分在计算Fourier系数时,常用分部积分的技巧。

\paragraph{与力学专业内容的联系}

微积分将伴随我们后续整个学习生涯,几乎没有哪一个力学专业课用不到微积分,我们只介绍一些典型应用。

\begin{itemize}
	\item 质心与转动惯量——重积分
	      \[
		      \symbfit{r}_C=\dfrac{\int_\Omega \symbfit{r}\,\mathrm{d}m}{\int_\Omega \mathrm{d}m}, \quad J_C=\int_\Omega r^2\,\mathrm{d}m.
	      \]
	\item 建立连续体受力微分方程——微元法

	\item 弹性力学中轴对称问题应力函数所满足的相容方程
	      \[
		      \left( \frac{\mathrm{d}^2}{\mathrm{d}\rho ^2}+\frac{1}{\rho}\frac{\mathrm{d}}{\mathrm{d}\rho} \right) ^2 \symit Φ =\frac{\mathrm{d}^4 \symit Φ}{\mathrm{d}\rho ^4}+\frac{2}{\rho}\frac{\mathrm{d}^3 \symit Φ}{\mathrm{d}\rho ^3}-\frac{1}{\rho ^2}\frac{\mathrm{d}^2 \symit Φ}{\mathrm{d}\rho ^2}+\frac{1}{\rho ^3}\frac{\mathrm{d} \symit Φ}{\mathrm{d}\rho}=0.
	      \]
	      ——Euler方程

	\item 信号频谱分析——Fourier级数
\end{itemize}

\paragraph{学习建议}

我们以后的运用以计算为主,如果初学时对一些严格定义有困惑,可以在不影响计算与运用的前提下适当放一放。在能进行计算的前提下,回头结合计算实例进行理解。

以计算为主不代表要去算各种奇奇怪怪的极限、积分。如果没有准备竞赛的要求或是出于个人兴趣爱好,没有必要算那些奇怪的积分(特别是有些来路不明的钓鱼题)。一来,能够计算出结果的积分是十分有限的,更可能要用到许多特殊函数的性质;二来,我们以后会用数值方法去处理广泛的积分,若非分析性质的需要,数值解是足够的。

一般微积分课程中对于矢量分析的内容是不足的,这里可以做一些额外补充,否则面对普通物理学当中的力学推导,以及理论力学当中的运动学部分的推导可能会比较吃力,推荐书目有\cite[托马斯大学微积分]{李伯民2009托马斯大学微积分} 的10、11两章,\cite[工程数学——矢量分析与场论]{谢树艺2015工程数学},\cite[微分几何]{彭家贵2002微分几何}的前两章。

\subparagraph{参考书目}\mbox{}

参考书选择大前提:一定要适合自己,不要三人成虎,先入为主地迷信或厌恶某本教材!学习过程中可广泛参考不同教材,但应该以其中一两本为主,其余为辅。

首先,同济大学的《高等数学》能够满足正常的微积分知识学习,包括很多高中参加物理竞赛的同学学习高数时也参考了这本书。这本书的例子、逻辑与习题量至少能满足力学专业的学习,如果要参考其他教材,请至少以此为下限进行参考。这套教材最大的问题是过于中庸,例如:虽然有例子但不够丰富,逻辑性与严密性卡在一个比较尴尬的位置,图示也不太丰富。

\begin{itemize}
	\item \cite[普林斯顿微积分读本]{杨爽2010普林斯顿微积分读本}

	      入门级微积分读本,适合没有相应数学基础的高中生以及哲、法、文学生阅读,主体内容是一元微积分的部分。这本书的特点是起点足够低,对于基础内容的讲解也足够细致,不一味强调严谨性,因而适合初学者入门。问题在于内容较浅、过于冗杂,这种风格并不适合一般理工科的后续学习和科研要求。

	\item \cite[托马斯大学微积分]{李伯民2009托马斯大学微积分}

	      入门级微积分读本,但基本涵盖了国内一般微积分教材的大部分内容。特点是起点低,知识点引入比较流畅,内容丰富,实例、图示与习题都很充足,也很适合初学者入门。

	\item \cite[工程数学——矢量分析与场论]{谢树艺2015工程数学}

	      这本书在微积分水平上对矢量分析进行了比较全面的讨论,第一章讨论了矢量函数的微积分,后边讨论了矢量场、梯度算子及一般的曲线坐标系中的运算,对于本科课程级别的运动学、弹性力学和流体力学中的矢量运算比较有帮助。

	\item \cite[微分几何]{彭家贵2002微分几何}

	      这本书所讲的第一部分是古典微分几何的内容,适合在学了微积分之后适当阅读。涉及的数学工具只有微积分、空间解析几何及少量的线性代数,理论不抽象,计算较复杂,同时,工程力学的研究大都是在经典的三维欧氏空间$\mathbb{R}^3$下的,所以这本书正适合力学专业的同学阅读。

	      古典微分几何大致可分为曲线论和曲面论。曲线论的研究方法可以追溯到对质点运动轨迹的研究,比较有意思的是标架方法,学过这一部分内容之后正好可以重新审视一下质点运动学的内容。曲面论的内容相对比较丰富,一般来说会先从曲面的基本形式和曲率入手,通过活动标架方法,研究曲面的结构方程,最后是曲面的内蕴几何学及测地线的理论。
\end{itemize}
\subsubsection{线性代数}

\paragraph{主体知识介绍}

线性代数以研究线性变换、线性方程组为主。在数学上,线性一般指的是齐次性和可加性,例如对于映射$f:\mathbb{R}^n\to\mathbb{R}$,这两条性质是

\begin{itemize}
	\item 齐次性:$f(a\symbfit{x})=a\cdot f(\symbfit{x}).$

	\item 可加性:$f(\symbfit{x}+\symbfit{y})=f(\symbfit{x})+ f(\symbfit{y}).$
\end{itemize}

这时就说映射$f$是线性的。研究在有限维线性空间(例如$n$维欧氏空间$\mathbb{R}^n$)中线性映射的代数性质的分支就是线性代数。那么典型的线性映射有哪些呢?我们举平面解析几何中的两个例子说明。

设平面上有一点$(x,y)$,现在将其绕原点逆时针旋转$\theta$度,利用三角函数的诱导公式容易得到变换后点$(x',y')$的坐标为
\[
	\begin{dcases}
		x'=x\cos \theta -y\sin \theta, \\
		y'=x\sin \theta +y\cos \theta .
	\end{dcases}
\]
不妨将点$(x,y)$看作是矢量$(x,y)$。在这里看来,齐次性就是在变换先后将向量$(x,y)$拉长若干倍;可加性可看成将向量$(x,y)$拆成$(x,0)$与$(0,y)$之和,将二者分别逆时针旋转$\theta$度后再加和,得到的结果与直接旋转向量$(x,y)$是一致的。所以,这种旋转变换是线性变换。

现在,来考察将点$(x,y)$平移$(a,b)$之后的结果,即
\[
	\begin{dcases}
		x'=x+a,  \\
		y'=y+b . \\
	\end{dcases}
\]
我们说这个变换不满足齐次性,也不满足可加性。仅以齐次性为例说明,将平移变换记为$f$,则$f(x,y)=(x+a,y+b)$,对于任意一个给定的$k$,$f(kx,ky)=(kx+a,ky+b)\ne (kx+ka,ky+kb)$。从几何直观上来看,先将向量$(x,y)$伸长$k$倍,再将箭头的位置平移$(a,b)$,与先将向量$(x,y)$箭头的位置平移$(a,b)$,再将其伸长$k$倍的结果显然是不一样的。类似也能验证可加性不成立。但是,如果我们将点$(x,y)$视为三维空间中的向量$(x,y,z)$,并规定新的平移变换为
\[
	\begin{dcases}
		x'=x+az, \\
		y'=y+bz, \\
		z'=z   . \\
	\end{dcases}
\]
能够验证,这样规定的平移变换又是满足齐次性和可加性的了,因而是线性变换。同时,$z$的选取是任意的,为了方便可以直接令$z=1$。

借由上边两个线性变换,我们还能够发现,如果已知线性变换的规则以及变换后的结果,返回去求变换前的点的过程,实际上就是解线性方程组。因此,解线性方程组也与线性变换紧密关联。

在这个阶段,我们通常只研究实数域内的线性变换。线性变换都能写成矩阵,矩阵是一张数表,我们再规定矩阵的基本运算规则,就能对线性变换进行研究了。矩阵除了加减、数乘,还有许多其他运算,如矩阵相乘、行列式、逆、转置、秩等等,这些运算是帮助我们研究线性变换的有力工具。

线性变换的作用对象是向量,所以我们也要研究向量、向量组和向量空间。向量组线性相关/线性无关的概念很常用,后续学习中它将反复出现。向量空间中最重要的概念是基向量,所以我们会研究基向量与向量空间的关联,以及如何在一个向量空间中“选出”一组比较好的基向量(Schmidt正交化)。特别的,我们会研究欧氏空间$\mathbb{R}^n$,它比一般的线性空间多了“内积”和“距离”这两个概念,而三维欧氏空间$\mathbb{R}^3$上又有“叉乘”这一运算,这些是研究空间解析几何的基本工具。

解线性方程组是线性代数很重要的一部分,不过事实上我们并不常用Cramer法则,更多的是用Gauss消元法(初等变换法)。通过计算低阶的线性方程组要掌握这套流程,至于高阶的线性方程组可以尝试编程实现,在后续的课程中也会讨论解大规模线性方程组的数值方法。

研究线性变换连续作用于向量若干次产生了对特征值与特征向量相关内容的研究。一些矩阵能够写成$\symbfit{P}^{-1}\symbfit{\symbfit\Lambda} \symbfit{P}$,其中$\symbfit{\symbfit\Lambda}$是对角矩阵,称这个过程为矩阵的相似对角化。根据矩阵乘法的结合律,问题转化为对角矩阵的幂的计算,而这是容易的。特征值与特征向量不仅能用于求矩阵的幂,还能用来求微分方程和差分方程,相似关系还可以描述同一个线性变换在不同基下的表示。此外,有一些矩阵具有特定的性质,如正交矩阵、对称矩阵等需要注意。

线性代数课程中一般还会介绍二次型理论。二次多项式可以用矩阵表示,而二次多项式与空间曲面密切相关,所以这部分内容可以与空间解析几何相联系。

除了以上这些基本知识,还可以做适当的补充,包括多项式、Jordan标准型以及线性空间。

\paragraph{与力学专业内容的联系}

应力、应变分析——特征值与特征向量问题

结构力学中的力法方程、有限元方程——求解线性方程组

多自由度系统模态分析——特征值与特征向量问题

\paragraph{学习建议}

在学习线性代数时,一个比较好的做法是寻找实例和做推广。例如,积分运算满足Cauchy-Schwarz不等式,而在欧氏空间部分,我们也有一个基于内积定义推导出的Cauchy-Schwarz不等式,由此可以想到,积分运算是不是某种意义上的内积呢?这种类比、抽象的思考方式在代数的学习中是比较重要的,不必拘泥于课本上的内容,最好要广泛联系,多多思考。无论如何,偏向代数的数学第一位是理解概念、寻找动机,然后才可能是做各种计算。

一般的线性代数课程所讲内容未免略显单薄。为此,最好补充关于多项式、Jordan标准型、线性空间的知识,以及线性代数的一些具体应用,推荐书目有\cite[Gilbert Strang\kern1em 线性代数:第5版]{线性代数5}、\cite[线性代数应该这样学]{阿克斯勒杜现昆2016线性代数应该这样学}、\cite[高等代数学]{张贤科2004高等代数学}。

\subparagraph{参考书目}\mbox{}

目前绝大多数国内的线性代数教材都采用开篇讲行列式的模式,稍好一些的教材会直接跟进解线性方程组的Cramer法则。线性代数的基本理论比较琐碎,然而行列式起手的这种讲法并不够合理,线性代数最核心、最重要的内容应当是线性变换,直接引入行列式很容易让初学者不知所云。所以,这里推荐三本引入较为自然的教材,目的是提纲挈领,先对线性代数形成一个大致印象,知道线性代数在研究什么。

\begin{itemize}
	\item \cite[普通高中课程标准实验教科书 数学 选修4-2 A版 矩阵与变换]{矩阵与变换}

	      目前绝大多数地区都已经不将这本高中教材作为高考备选选项,但这本教材通过平面解析几何引入线性变换和矩阵的做法是很好的,用线性变换将整个线性代数串了起来,思路十分清晰。不足之处是主要只讨论了二阶矩阵的理论,所以不太适合完整的线性代数的学习。
	\item \cite[邱志鹏、杨建新、钱雄平\kern1em 线性代数]{线性代数}

	      这本教材是少有的从线性变换讲起的国内教材,引入矩阵也比较自然。并且还有一个亮点是这本书在附录部分给出了平面解析几何中线性变换的几何含义。尽管在引入部分做得很好,但总的知识量与国内其他教材并无差异。

	\item \cite[Gilbert Strang\kern1em 线性代数:第5版]{线性代数5}

	      并不是说从解线性方程组起讲的教材都不好,这本教材就是从线性组合和解线性方程组引入的,引入也很自然、直观。同时,这本书中包含了奇异值分解、复矩阵以及其他线性代数的应用,如快速Fourier变换、最小二乘近似、图像处理等内容,就实用性这一点来说是很棒的。

	      线性代数的进阶便是矩阵理论,所以,其他的推荐书目我们都放在矩阵理论部分。
\end{itemize}
\subsection{进阶数学培养}

在学习过微积分和线性代数后,一些学校会选择在本科阶段开设进一步的数学课程,以满足专业学习的需要,这里介绍最基本的几门数学课程。

\subsubsection{概率论与数理统计}

\paragraph{前置知识:}微积分

\paragraph{部分知识介绍}

条件概率的Bayes公式,建立先验概率与后验概率的联系。

常用的随机变量分布及其数字特征。常用的离散分布有二项分布、Poisson分布、几何分布,连续分布有均匀分布、指数分布、正态分布、Weibull分布。数字特征则是期望、方差和协方差。

数理统计当中的几个分布和参数估计。重要的分布有$\chi^2$分布、$t$分布、$F$分布。参数估计主要是矩估计和最大似然估计。

\paragraph{与力学专业内容的联系}

零件可靠性/寿命分析——Weibull分布及特征量

随机振动——随机过程的统计特征

\paragraph{学习建议}

对于大部分做力学相关的同学来说,用到概率论与数理统计的情形其实不多,能应付考研即可。笔者也很难给出合适的建议,最好还是要熟悉基本的运算、了解一些常用的概率分布、其应用场合以及基本的统计方法,这可能在大规模的数据处理方面派上用场。

\subsubsection{复变函数与积分变换}

\paragraph{前置知识:}微积分

\paragraph{部分知识介绍}

复变函数指的是函数的定义域和值域都是复数的函数,这门课上只研究自变量是一个复数的情形。一个复数$z\in\mathbb{C}$可以写成$z=x+\mathrm{i}y, x,y\in\mathbb{R}$的形式,也就是说复变函数与二元实函数很像,可以把这门课当作是二元微积分的升级版,但由于复数又不同于一般的二维向量,所以复变函数这里会有更多的结论。

首先要熟悉复数的基本运算,复数与二维欧氏空间中的向量十分相似,不仅可以用平面表示,向量可以进行的所有运算复数也可以进行。但复平面$\mathbb{C}$与二维平面$\mathbb{R}^2$最重要的区别是复数上有“真正的”乘法,注意一般的$\mathbb{R}^2$上只定义了数乘:$\mathbb{R} \times \mathbb{R}^2 \to \mathbb{R}^2$和内积:$\mathbb{R}^2 \times \mathbb{R}^2 \to \mathbb{R}$,因为这种映射的“乘数”和“结果”形式不同,所以说它们不是“真正的”乘法,而不曾定义$\mathbb{R}^2 \times \mathbb{R}^2 \to \mathbb{R}^2$这样的“真正的”乘法。在复平面上则有满足$\mathbb{C} \times \mathbb{C} \to \mathbb{C}$的乘法,这是因为复数比一般的二维平面多了$\mathrm{i}^2=-1$这样一个定义,这就导致复变函数的推论远比一般的二元实函数要多。

认识了基本的复变函数后,首先要讨论的就是可导/可微的概念,这里我们最关注的是一类被称为解析函数的复变函数。一般的复变函数$f$是可以看成两个关于$x$,~$y$的实函数的,而注意$z,\bar{z}$(即$z$的共轭)是线性无关的,所以一般的复变函数可以看成是关于$z$,~$\bar{z}$的二元函数,解析这件事说的其实就是在定义域上,$f$对$z$可导,同时与$\bar{z}$无关,后者就是Cauchy-Riemann条件:
\[
	\frac{\partial f}{\partial \bar{z}}=0 \Leftrightarrow
	\begin{dcases}
		\dfrac{\partial u}{\partial x}=\dfrac{\partial v}{\partial y},  \\
		\dfrac{\partial v}{\partial x}=-\dfrac{\partial u}{\partial y}. \\
	\end{dcases}
\]
解析函数的性质很好:无穷次可微、实部与虚部分别满足调和方程,即设$f(z)=u(x,y)+\mathrm{i}v(x,y)$,则
\[
	\begin{dcases}
		\dfrac{\partial ^2u}{\partial x^2}+\dfrac{\partial ^2u}{\partial y^2}=0, \\
		\dfrac{\partial ^2v}{\partial x^2}+\dfrac{\partial ^2v}{\partial y^2}=0. \\
	\end{dcases}
\]
这直接与一类偏微分方程建立了联系。由于描述单个复变量就已经要用到二维平面,那么描述复变函数就用到四维的空间,所以没法用三维坐标系直观地描述函数图像。一种办法是先在复平面上画上若干正交的直线组,考虑经过一个函数作用后,这些直线将变成何种形式的曲线。这一部分内容可进一步去讨论保形映射的相关内容。

复变函数的级数部分讲的主要是Taylor级数和Laurent级数。Taylor级数与实函数部分差别不大,但是由此直观地审视了“收敛圆”的概念,并且考察复变函数对考虑实函数级数的收敛域是有帮助的。一个经典的例子是$f(x)=\dfrac{1}{1+x^2}$,在实轴上看该函数处处都是任意阶可导的,而在复平面上看,它有两个极点$\pm\mathrm{i}$,于是在$0$处展开时该级数的收敛半径为$1$。Laurent级数是对Taylor级数的推广,允许级数中存在负幂次项。对于极点数目有限的函数,Laurent级数至少能够对一个圆环区域收敛,这就允许我们考察在极点处的函数展开,例如余割函数$\csc z$在$z=0$处的展开:
\[
	\csc z=\frac{1}{z}+\frac{z}{6}+\frac{7z^3}{360}+\cdots\,.
\]
Z变换与Laurent级数紧密相连,Z变换在处理离散序列信号,以及解差分方程中发挥重要作用。

复变函数的积分部分的内容十分精彩,可以说复变函数成为一个相对独立的数学分支与这一部分的结论分不开。借助Green公式,能够证明Cauchy积分公式和闭路变形定理,将解析复平面上的环路积分归结成极点附近积分的计算,这里会考察极点的性质以及留数。随后还有将导数与积分建立起联系的Cauchy积分公式,这为级数中系数的计算提供了另一种手段。

有些教材会在后边附上关于Fourier变换和Laplace变换的内容,也有课程安排将这两部分放到数学物理方程中去讲。所谓积分变换,是将一个函数$f(t)$通过一个变换$\mathscr{T}$变成另一个函数$F(p)$的过程:
\[
	F\left( p \right) =\mathscr{T}\left[ f\left( t \right) \right] =\int_a^b{f\left( t \right) K\left( t,p \right) \,\mathrm{d}t}.
\]
而新函数$F(p)$中也包含了原函数$f(t)$的信息,从而将对$t$研究转换成对$p$的研究。根据所选的变换核函数$K(t,p)$的形式,将积分变换分成不同种类,如选择$K(t,\omega)=\mathrm{e}^{-\mathrm{i}\omega t}$得到Fourier变换,选择$K(t,s)=\mathrm{e}^{-st}$得到Laplace变换等。

Fourier变换是从Fourier级数引入的,Fourier级数考察的是周期函数中一个有限的区间, Fourier变换考察的是非周期函数,而非周期函数可以看作是周期无穷大的周期函数,所以能够通过Fourier级数引入Fourier变换。不同学科中对Fourier变换的定义是略有区别的,有些地方可能差一个常数倍,讨论之前需要注意。如果不引入广义函数$\symup\delta(t)$,像$1$,~$\sin t$,~$\cos t$这样简单的函数都是没法做Fourier变换的,但是注意,这里边的广义函数$\symup\delta(t)$其实并不是函数,由于所学知识有限,不会对其做过多的探讨,但是一定注意其与我们传统意义上的函数之间的区别。例如,涉及到$\symup\delta(t)$的导数的一定是弱导数,而不是经典意义下的导数,在使用时务必要遵循我们事先约定好的定义。Fourier变换的重要应用意义在于求解微分方程以及信号的频谱分析。

Laplace变换是将算子法解微分方程严密化而来的。Laplace变换没有像Fourier变换那么多需要注意的坑,记好基本内容就够用了。这里多提一点Fourier变换和Laplace变换为何能在解微分方程中发挥作用,这是这两种变换的微分与积分性质决定的,这两种变换能够将微积分这两种分析学运算转化成乘除法这样的代数运算,从而在形式上简化了求解过程。那么,问题一定变得简单了吗?不一定,因为这个过程实际上将复杂的计算转移到了求变换与逆变换的过程中了,如果有完善的积分变换表可以查询,那么问题确实简单不少。这个过程就好像人们曾经为了简化大数乘法而发明对数,将乘法转化成加法一样,形式上问题简化了,但是复杂的计算被隐藏到了求对数这一过程中。

\paragraph{与力学专业内容的联系}

离散信号分析、差分方程求解——Z变换,实际上就是Laurent级数

连续介质力学平面问题中,关于有势场散度的描述——解析函数的实部和虚部均为调和函数

二维弹性问题求解、无限大平板圆孔问题、断裂力学三类基本裂纹求解——弹性问题的复变函数解法

积分变换解微分方程、积分方程——以Fourier变换和Laplace变换为主

\paragraph{学习建议}

复变函数部分,类似微积分,以熟悉基本的运算规则和定理公式为主,但没必要做太多奇怪的计算,我们直接用到复变函数计算的内容其实也不多。重点可以放在积分变换部分,主要去熟悉积分变换的基本性质,为后续解微分方程服务。Fourier变换可以额外留意一下,实验手段或是一些特定问题的研究中,Fourier变换是十分常用的,务必要将所有概念牢牢掌握。

\subparagraph{参考书目}\mbox{}

\begin{itemize}
	\item \cite[工程数学——复变函数、积分变换与场论]{工程数学——复变函数、积分变换与场论}

	      这本教材的优势在于对于前两章的基础处理比较细节,在此处也直接引入了对保形映射的讨论,几何直观性做的比较好。此外,在积分变换部分,这本书给出了对Mellin变换、Hankel变换和Z变换的内容,这些变换在特定的场合能够发挥作用,可以作为补充材料有针对性地学习。
	\item \cite[复分析:可视化方法]{needham2009复分析可视化方法}

	      这本教材从几何直观的角度讲解复变函数,同时内容没有超出我们一般所学的复变函数的内容太多,可以满足几何可视化的需求。该书内容较多,不必通读,建议有选择、有针对性地读。
\end{itemize}
\subsubsection{数学物理方程}

\paragraph{前置知识:}微积分、线性代数、积分变换

\paragraph{部分知识介绍}

这门课的内容概括起来就是:如何解二阶线性偏微分方程的定解问题。

首先会了解一些偏微分方程中的基本概念,以及从一些常见的数学物理问题中提炼出不同的二阶线性偏微分方程,其中一些具有共同形式,于是将其大致分类为“椭圆型”、“抛物型”与“双曲型”:

\begin{itemize}
	\item \textbf{椭圆型}:$\Delta u = f$.

	\item \textbf{抛物型}:$\dfrac{\partial u}{\partial t}-k\Delta u = f$.

	\item \textbf{双曲型}:$\dfrac{\partial^2 u}{\partial t^2}-\alpha^2\Delta u = f$.
\end{itemize}

其中$\Delta$是Laplace算子,$u$是未知函数。这个名字由来与对曲面上一点的类型相似,都来源于对特征值的讨论。不同类型的方程具有不同的性质,分门别类地讨论是很有好处的。 %%%%附图

解偏微分方程与解常微分方程的不同之处在于,假设方程的形式比较好,可以直接积分,那么常微分方程做一次积分会出现一个待定的任意常数,而偏微分方程做一次积分会出现一个待定的任意函数。所以对偏微分方程作积分最多能得到形式上的解,有的时候可以根据初始条件或边界条件确定解的具体形式,然而更多的时候是没法确定的,退一步讲有些形式解还包含复数,从而引起很复杂的讨论,更何况能直接作积分的偏微分方程更是少之又少。

处理线性偏微分方程的手段一般有三种:级数法(分离变量)、积分变换法和Green函数法,这些方法能用的原因与线性方程解的可加性离不开。

级数法首先假设解函数可以写成若干一元函数之积的形式,如对于$u=u(x,y)$,通常假设$u(x,y)=X(x)Y(y)$(能够证明这样假设是合理的),随后通过选择一个合适的函数列(一般而言是一个可列无穷多的序列),将解表示为这个函数列的线性组合。最常用的函数列是Fourier级数,但对于不同形状的区域来说,这个函数列的选择一般是不同的。至于这个函数列如何选取,还要用到Sturm–Liouville方程的理论,同时涉及到一些特殊函数的知识。

从Fourier级数引入Fourier变换的思想也可以用在这里。级数法处理的都是有限区域,对于半无界或无解区域就要用到积分变换法了。并且类似级数法,当使用不同形状的坐标系考察问题时,用到的积分变换可能也是不同的。

Green函数法具有深厚的物理背景,它从考虑点电荷场分布的问题而来。直观地说,如果我们知道点源的分布情况,就能通过对研究区域进行线性叠加,找到一个满足给定定解条件的方程的解,这个点源的分布函数就是Green函数。所以,如果能找到某一个问题相应的Green函数,这个问题就好解决了。Green函数不仅与方程的形式有关,与边界条件也有关,甚至与问题的维数也有关。

\paragraph{与力学专业内容的联系}

诸多椭圆型微分方程的求解(弹性力学中的静力学问题、模态分析等)——基于分离变量法的Fourier级数解法、边界元法方程构建——Green函数法

\paragraph{学习建议}

一般来说,这门课的起点不会太高,甚至有些学校的数学物理方程这门课被称为“微积分习题课”都不夸张,但其实如果能够多学一点,这门课还是很实用的。这门课的一大特点是,理论不复杂,看起来好像很好处理,但其中涉及到的细节很多,所以最好将学习过程中涉及到的计算、推导流程操作一遍。

不过,得益于各类计算机硬件条件和数值计算方法的发展,以及出于对非线性方程求解的需求,求解析解似乎变成了一个有些费力不讨好的事情。但不论如何,解析求解都应该是理论工作者应当追求的东西。

\subparagraph{参考书目}\mbox{}
\begin{itemize}
	\item \cite[吴崇试\kern1em 数学物理方法(第3版)]{数学物理方法(第3版)}


	      这本书的内容十分丰富,前一部分非常详尽地讨论了复变函数、积分变换和特殊函数的理论,这一部分也可以作为复变函数部分学习的参考资料。后一部分则是解线性偏微分方程的内容,相比于大多数国内教材,这本教材将数学方程与物理背景的联系比较好,对于各类具体形式的方程及其边界条件的讨论也比较详尽。此外,这本书还简单地介绍了变分法,变分法在理论分析和计算上的用处都很大,后边还会介绍。

	\item \cite[特殊函数概论]{特殊函数概论}

	      特殊函数的研究内容十分宽泛,但我们只需要关注在解一部分方程的过程中出现的特殊函数就足够了。这本书的一二章介绍了级数和二阶常微分方程的基本理论,第三、五、七章分别介绍了$\Gamma$函数、Legendre函数和Bessel函数这几类在物理方程中经常出现的特殊函数。由于处理特殊函数的技巧性很强,所以在数值解出现之后人们往往也就不那么关注特殊函数解了,但在有些时候分析问题的性质时还是有用的。这本书本身可以作为工具书使用,查询一些特殊函数的特定性质。

	\item \cite[积分方程(第3版)]{积分方程(第3版)}

	      讲授积分方程求解的课程和参考书很少,这本书是少有的讲积分方程求解的参考书。这本书没有用到太高深的数学概念,积分方程不仅有很多实际应用的情景,并且学习积分方程也可以为泛函分析提供学习的实例与动机,如果有精力,应该好好看看这本书。

	\item \cite[微分方程与数学物理问题]{ibragimov2013微分方程与数学物理问题}

	      这本书的前五章大致是经典的处理线性微分方程的理论,其中的亮点是物理模型比较丰富,以及介绍了一阶偏微分方程的理论。后半部分从对称性(Lie群)的工具探讨了非线性方程的处理方法,对用解析手段处理微分方程感兴趣的读者可以将其作为拓展阅读材料。

	\item \cite[量纲分析与 Lie 群]{孙博华2016量纲分析与Lie群}

	      量纲分析是处理力学问题当中常用的一种技巧,而量纲分析与Lie群存在联系。这本书还比较细致地介绍了用Lie群工具处理微分方程的基本理论和具体算法。另外,这本书中涉及到的物理模型也是比较丰富的。
\end{itemize}

\subsubsection{数值分析}

\paragraph{前置知识:}微积分、线性代数

\paragraph{部分知识介绍}

很多问题的解析求解是十分困难的,甚至是不可能的,所以在大多数时候,我们只能退而求其次,去寻求近似解。同时,出于计算机求解的需要,发展各种离散的近似算法就是十分必要的。这就是数值分析所做的工作,除了构建算法,还需要研究这些算法的可靠性,具体地说,有算法效率、稳定性分析、误差分析。

数值分析课程中一般包含以下内容:非线性代数方程求解、线性方程组求解、插值与逼近、数值积分、矩阵特征值求解和常微分方程求解。这门课程的整体特点是内容比较碎,不同模块之间的关联不大。

三次、四次方程的解的形式已经十分复杂,而五次及以上方程和一般的超越方程更是没有根式解,所以求一般的非线性代数方程的近似解就很必要了。非线性代数方程的求解方法中,Newton迭代法及其各种改进是比较重要的,这在优化方法中也会大量使用。

大规模的线性方程组按Gauss消元求解,它可以归结为直接分解法,另一种是迭代法。它们的构造与分析主要用到矩阵理论,大致包括矩阵分解和范数计算。一般情况下,直接分解法的时间复杂度与Gauss消元相当,而迭代法则是压缩映射原理的直接应用。特别的,分解法中有针对正定对称阵的Cholesky分解和针对三对角方程的追赶法,时间效率有所提高。

插值、逼近和曲线拟合的概念比较相似,但又有一定的区别。插值、逼近和拟合都是是用一组函数的线性组合来近似一个函数关系,区别在于,插值要求这个近似通过给定的数据点;逼近要求这个近似最接近给定的函数关系;拟合要求这个近似最接近给定的数据点。%%%%插图
插值要求近似函数通过给定数据点,这是比较好理解的,问题的难点在于估计误差;逼近和拟合中有一个描述词——“最接近”,在此以前,我们接触的描述接近程度的工具,也就是距离,都是针对点来说的,而此处的距离则是针对函数提出的。这里概念的推广可能需要转换一下脑筋,将函数视为一种向量,而一组基函数张成一个线性空间,我们就在这个线性空间上讨论问题,这样,研究对象就很接近代数的内容了。

数值积分的方法十分朴素,其思想就是还原了用黎曼和定义积分的方法,因此这一部分就不展开介绍太多了。只提两个比较有用的算法,一是Gauss求积方法,这种方法的代数精度比较好,因此应用较多;二是Romberg算法,它能加快收敛速度,且具有稳定性好的优点,同时这种思想也能推广到一些其它的数值计算方法上,此处可多多留意。

矩阵特征值的求解没有什么需要强调的。一个比较有意思的点是,求解多项式方程时,因为Frobenius矩阵的特征多项式是首一多项式,从而可将多项式求根的问题转化成特征值求解问题,并且该方法能得到多项式的所有根,而不是像迭代法一样只能求得一个根。

解常微分方程最常用的方法是Runge-Kutta方法,该方法的效果一般都比较好,目前已经在很多地方得到应用。

\paragraph{与力学专业内容的联系}

可以说,如今的力学有很大的一部分是在做计算,所以数值方法与力学问题的求解是广泛相联系的。这里只举一些例子:

\begin{itemize}
	\item 数字信号分析——快速Fourier变换

	\item 模态分析的数值解法——矩阵特征值的数值解

	\item CAD当中的B\'ezier曲线——基于Bernstein多项式的插值

	\item 有限元法中形函数的构建——Lagrange插值

	\item 有限元法中对积分的处理——Gauss积分公式

	\item 有限元法方程求解——Cholesky分解解线性方程组
\end{itemize}

\paragraph{学习建议}

数值分析这门课不只要学习理论,更要注意编程实现一些重要的算法。尽管诸如MATLAB等数学软件中已有现成的函数可以直接调用,但在初次学习时还是应该手动编程实现整个过程,一方面加深对算法的理解,另一方面也可以锻炼自身的编程能力。

此外,可以阅读[28]的前四章,这一部分讲了数值分析中一些来源于泛函分析中的概念,可以了解一些概念的准确定义,将一些问题纳入线性空间下直观地考虑。

\subparagraph{参考书目}\mbox{}

\begin{itemize}
	\item \cite[李庆扬\kern1em 数值分析]{李庆扬2009}

	      这本书的内容相当丰富,并且内含一些特别实用的算法,如快速Fourier变换等。不过该书没有代码示例,可以参考一些算法实现集,或者在互联网上寻找。
\end{itemize}

\subsection{后续及定向数学培养简介}

上述进阶培养的内容对于本科阶段的专业课来说基本够用了,但是也有一些针对特定方向,以及满足更深入学习需求的数学课。因为这些内容已经超出新生所能接触到的数学很多了,所里这里我们只简单介绍一下,更详细的信息请有兴趣或需要的读者自己查阅资料。

\paragraph{常微分方程}

常微分方程在微积分中讲过一些内容,在此基础上,可以学习这些内容:一阶微分方程解的性质、非线性微分方程的定性分析、高阶方程的级数解、一阶线性偏微分方程和边值问题。

对一阶微分方程解的性质的研究为数值方法提供了理论基础;一阶线性偏微分方程和边值问题可以选择在开设数学物理方法之前学习一下,这些内容在数学物理方法中有直接的应用;非线性微分方程的定性分析的内容十分丰富,其中关于稳定性的理论在高等动力学及控制原理中有运用。

\subparagraph{参考书目}\mbox{}

\begin{itemize}
	\item \cite[王高雄\kern1em 常微分方程(第四版) ]{王高雄2020常微分方程}
\end{itemize}

\paragraph{矩阵分析}

线性代数部分的理论和计算实际上是不太够用的,一些比较关键的问题处理的概念思路可能也是要在矩阵分析中才能遇到。在标准型这里一定要掌握Jordan标准型的计算方法了,矩阵函数和后边张量函数的计算都会用到。矩阵分解、矩阵范数都是相当常用的概念,其在数值分析、数据处理等很多方面都有应用。在此之前,我们考虑过的导数都是对标量的,从这门课开始将会考虑对矢量或矩阵求导,为张量分析做准备。

矩阵分析中实际上没有用到太深入和抽象的数学概念,其中的概念与定义还是比较直观的,因此首先还是要理解概念。其次,矩阵分析中的计算很多,计算量也比较大,因此在学习时要亲自动手计算。另外,矩阵分析是相当实用的,还可以与一些具体应用或数值方法结合起来。

\subparagraph{参考书目}\mbox{}

\begin{itemize}
	\item \cite[史荣昌\kern1em 矩阵分析]{史荣昌2010}
	\item \cite[线性代数高级教程——矩阵理论及应用]{线性代数高级教程}
	\item \cite[线性代数应该这样学]{阿克斯勒杜现昆2016线性代数应该这样学}
	\item \cite[高等代数学]{张贤科2004高等代数学}
\end{itemize}

\paragraph{张量分析}

张量分析是力学系同学应该掌握的一项技术。如果你学习过徐芝纶先生所写的《弹性力学》,就会感到方程的推导实在是过于繁杂和琐碎。使用张量语言的初衷,就是为了简化方程推导。

学习分析部分前,我建议先学习一下局部微分几何(参考书目 \cite[微分几何]{彭家贵2002微分几何}),这样能更快接受指标规则,也能在学习曲线、曲面上的张量分析时容易很多。

学完张量分析,我们可以回头审视建立弹性力学、流体力学基本理论的过程,尝试用张量语言完成这一过程。另一方面,可以看一些塑性力学甚至连续介质力学等内容了。需要指出的是,张量语言的作用是简化复杂的运算,这是对于建立和理解理论体系来说的,最后仍然绕不开方程或方程组的求解。

\

\subparagraph{参考书目}

\begin{itemize}
	\item \cite[现代张量分析及其在连续介质力学中的应用]{谢锡麟2014现代张量分析及其在连续介质力学中的应用}
	\item \cite[张量分析]{黄克智2003张量分析}
\end{itemize}

\paragraph{偏微分方程数值解}

\subparagraph{参考书目}

\begin{itemize}
	\item \cite[微分方程数值解法(第四版)]{李荣华2009}
	\item \cite[偏微分方程数值解法(第三版)]{陆金甫2016偏微分方程数值解法}
\end{itemize}

\paragraph{应用泛函分析}

\subparagraph{参考书目}\mbox{}

\begin{itemize}
	\item \cite[薛小平\kern1em 应用泛函分析]{薛小平2012应用泛函分析}
	\item \cite[柳重堪\kern1em 应用泛函分析]{柳重堪1986}
	\item \cite[变分法基础]{老大中2015变分法基础}
	\item \cite[数值泛函及其应用]{张维强2021}
\end{itemize}
% \textbf{[25] 薛小平,张国敬,孙立民,武立中.应用泛函分析(第三版)[M].哈尔滨:哈尔滨工业大学出版社,2012.}

% \textbf{[26] 柳重堪.应用泛函分析[M].北京:国防工业出版社,1986.}

% \textbf{[27] 老大中.变分法基础(第3版)[M].北京:国防工业出版社,2015.}

% \textbf{[28] 张维强,冯纪强,宋国乡.数值泛函及其应用[M].北京:科学出版社,2021.}

\subsection{其他参考资料}

\begin{itemize}
	\item \cite[力学导论]{杨卫2020}
	\item \cite[力学史]{武际可2010}
	\item \cite[中国力学学科史]{中国力学学会编著2012中国力学学科史}
	\item \cite[斯米尔诺夫高等数学(5卷11册)]{斯米尔诺夫2018斯米尔诺夫高等数学}
	\item \cite[数学指南:实用数学手册]{埃伯哈德·蔡德勒2012数学指南}
\end{itemize}

% \textbf{[29] 杨卫,赵沛,王宏涛:力学导论[M].北京:科学出版社,2020.}

% \textbf{	[30] 武际可.力学史[M].上海:上海辞书出版社,2010.
% }

% \textbf{	[31] 中国力学学会.中国力学学科史[M].北京:中国科学技术出版社,2012.
% }

% \textbf{	[32] (苏)斯米尔诺夫著.斯米尔诺夫高等数学(丛书,共5卷11册)[M].斯米尔诺夫高等数学编译组译.哈尔滨:哈尔滨工业大学出版社,2018.
% }

% \textbf{	[33] (德)E. Zeidler等著.数学指南——实用数学手册[M].李文林等译.北京:科学出版社,2012.
% }% \end{document}